\documentclass[11pt,xcolor={dvipsnames},hyperref={pdftex,pdfpagemode=UseNone,hidelinks,pdfdisplaydoctitle=true},usepdftitle=false]{beamer}
\usepackage{presentation}

% Enter presentation title to populate PDF metadata:
\hypersetup{pdftitle={How Informative are Job Posting Skill Measures? Evidence from Selection Decisions}}

% Enter path to PDF file with figures:
\newcommand{\pdf}{figures.pdf}

\begin{document}

% Enter title:
\title{How Informative are Job Posting Skill Measures?\\
Evidence from Selection Decisions}

\information
%
% Enter URL to research paper (can be commented out):
%[https://github.com/nikhilg/job-postings-paper]
%
% Enter authors:
{Nikhil George, Ramayya Krishnan, Rahul Telang}
%
% Enter location and date (can be commented out):
{Carnegie Mellon University}

\frame{\titlepage}

% Enter content of presentation:

\begin{frame}
\frametitle{Motivation}
\begin{itemize}
\item \textbf{Skills-first hiring revolution}
  \begin{itemize}
  \item Google, IBM, Walmart prioritize capabilities over credentials (Fuller et al., 2022; WEF, 2020)
  \item Reducing emphasis on credentials in favor of capabilities
  \end{itemize}

\item \textbf{Critical yet unproven infrastructure}
  \begin{itemize}
  \item Powers algorithms (LinkedIn, Indeed), HR analytics, policy decisions
  \item Assumed to reflect skill demands, but \al{no validation against outcomes}
  \end{itemize}

\item \textbf{Stakes}
  \begin{itemize}
  \item If uninformative: Undermines billion-dollar HR tech ecosystem
  \item If valid: Unlocks precision in labor markets
  \end{itemize}
\end{itemize}
\end{frame}

\begin{frame}
\frametitle{The Core Gap}
\begin{itemize}
\item \textbf{Skepticism}
  \begin{itemize}
  \item Postings may be aspirational, templated, or inflated with "nice-to-have" skills
  \item Perhaps postings reflect ideal candidates rather than minimal requirements
  \end{itemize}

\item \textbf{Validation imperative}
  \begin{itemize}
  \item Firms invest in parsing job text—but is it meaningful?
  \item Research relies on posting data without outcome grounding
  \end{itemize}

\item \textbf{Critical knowledge gap}
  \begin{itemize}
  \item Do skill measures derived from job postings predict actual hiring decisions?
  \item How informative are postings relative to traditional HR metrics?
  \end{itemize}
\end{itemize}
\end{frame}

\begin{frame}
\frametitle{Research Questions}
\begin{enumerate}
\item \textbf{Validation}: Can skill distance measures built on job postings be validated against selection decisions?

\item \textbf{Informativeness}: How does posting-derived information compare to traditional HR metrics?

\item \textbf{Application}: Can analytics built on job postings open up new insights about human capital management?
\end{enumerate}
\end{frame}

\heading{Research Contribution}

\begin{frame}
\frametitle{First Empirical Validation of Job Posting Informativeness}
\begin{itemize}
\item \textbf{Novel approach}
  \begin{itemize}
  \item Skill distance metric derived \al{solely from job text}
  \item Validated against \al{actual selection decisions} (1,370 internal applications)
  \end{itemize}

\item \textbf{Unique dataset}
  \begin{itemize}
  \item IT division of major U.S. financial institution
  \item 8,180 job vacancy postings (2018–2021)
  \item 650 selections, 720 rejections with complete histories
  \end{itemize}

\item \textbf{Key findings}
  \begin{itemize}
  \item \alg{84\% higher} selection probability for closest skill matches
  \item \alg{70\%} of multi-applicant vacancies select the closest candidate
  \item Job text outperforms traditional HR metrics (AUC \alg{0.62} vs. \alr{0.50})
  \end{itemize}
\end{itemize}
\end{frame}

\begin{frame}
\frametitle{Literature: A Gap in Validation}
\begin{itemize}
\item \textbf{Taxonomy-based approaches} (O*NET, ESCO)
  \begin{itemize}
  \item Static, subjective, misses contextual nuance
  \item Limited by predefined skill categories
  \end{itemize}

\item \textbf{Collaborative filtering methods} (LinkedIn)
  \begin{itemize}
  \item Co-occurrence $\neq$ predictive validity
  \item Learns from patterns, not outcomes
  \end{itemize}

\item \textbf{Credential-based proxies}
  \begin{itemize}
  \item Losing relevance in skills-first hiring
  \item Coarse-grained, indirect measures
  \end{itemize}
\end{itemize}

\begin{block}{Our Breakthrough}
\begin{itemize}
\item \textbf{Outcome-grounded validation} links text to real decisions
\item \al{Not just what skills are listed, but which ones actually matter}
\end{itemize}
\end{block}
\end{frame}

\begin{frame}
\frametitle{Research Setting \& Data}
\begin{itemize}
\item \textbf{IT division of major U.S. financial institution}
  \begin{itemize}
  \item Technology roles with precise skill requirements
  \item Structured internal mobility process
  \end{itemize}

\item \textbf{Unique proprietary dataset}
  \begin{itemize}
  \item 8,180 job vacancy postings (2018–2021)
  \item 1,370 internal applications
    \begin{itemize}
    \item 650 selections
    \item 720 rejections
    \end{itemize}
  \item Complete application histories
  \item Linked current and target job descriptions
  \end{itemize}
\end{itemize}
\end{frame}

\begin{frame}
\frametitle{Sample Job Postings Comparison}
\begin{center}
\fbox{\parbox{0.8\textwidth}{
\centering
\textbf{Target Position:} \\
Senior Data Architect
}}
\end{center}

\vspace{0.5em}

\begin{columns}
\begin{column}{0.48\textwidth}
\begin{block}{Close Match (selected)}
\begin{itemize}
\item \textbf{Position:} Data Engineer
\item \textbf{Shared skills:} Data modeling, ETL processes, SQL
\item \textbf{Key distance drivers:} Limited enterprise architecture experience
\end{itemize}
\end{block}
\end{column}
\begin{column}{0.48\textwidth}
\begin{block}{Distant Match (not selected)}
\begin{itemize}
\item \textbf{Position:} Full-Stack Developer
\item \textbf{Missing critical skills:} Data warehousing, dimensional modeling
\item \textbf{Different domain emphasis:} UI/UX vs. backend data systems
\end{itemize}
\end{block}
\end{column}
\end{columns}
\end{frame}

\begin{frame}
\frametitle{Conceptual Framework}
\begin{itemize}
\item Job descriptions encode skill requirements that matter for selection
\item Selection decisions reveal which aspects of postings are most important
\item Skill alignment between current and target roles predicts outcomes
\end{itemize}

\begin{block}{Formalized}
\begin{itemize}
\item When an applicant seeks a position that is 'distant' in required skills from their current job, they have lower probability of being selected
\item Selection probability decreases with skill distance
\item This relationship validates job posting content as meaningful
\end{itemize}
\end{block}
\end{frame}

\heading{Methodology}

\begin{frame}
\frametitle{Methodological Pipeline}
\begin{center}
\fbox{\parbox{0.8\textwidth}{
\centering
\textbf{[INSERT FIGURE 1 FROM PAPER]} \\
Pipeline showing embeddings → dimension reduction → metric learning → validation
}}
\end{center}

\begin{enumerate}
\item \textbf{Embeddings:} SBERT captures semantic richness of job text
\item \textbf{Dimension Reduction:} PCA + Fuzzy C-Means clustering
\item \textbf{Metric Learning:}
\begin{itemize}
\item Reweights dimensions (Mahalanobis matrix \al{M}) to reflect \al{what the firm historically valued}
\item Example: Stretches "Hadoop" dimension if critical for past hires
\end{itemize}
\item \textbf{Validation:} AUC 0.62 on held-out data
\end{enumerate}

\begin{block}{Key Nuance}
\begin{itemize}
\item \textbf{Maximizing informativeness:} Generic embeddings (ubiquitous) + firm-specific outcomes (rare)
\item Without selection data, embeddings remain noisy; with it, we isolate \al{actionable signals}
\end{itemize}
\end{block}
\end{frame}

\begin{frame}
\frametitle{Metric Learning: Optimization}
\begin{align*}
\text{Distance between jobs } j_c \text{ and } j_v: \\
d(j_c, j_v; M) = \sqrt{(j_c - j_v)^T M (j_c - j_v)}
\end{align*}

\begin{block}{Optimization Problem}
\begin{align*}
\text{Maximize: } & \sum_{(c,v): s_{c,v}=0} d(j_c, j_v; M) \\
\text{Subject to: } & \sum_{(c,v): s_{c,v}=1} d(j_c, j_v; M)^2 \leq 1 \\
& M \succeq 0
\end{align*}
\end{block}

\textbf{Intuition:} Learn distance metric that maximizes distance between non-selected pairs while constraining distance between selected pairs (Xing et al. 2002)
\end{frame}

\begin{frame}
\frametitle{Metric Learning: Strengths and Limitations}
\begin{columns}
\begin{column}{0.48\textwidth}
\begin{block}{Key Strengths}
\begin{itemize}
\item \textbf{Domain-Specific Adaptation}
\begin{itemize}
\item Tailors generic embeddings to firm's unique context
\item Explains performance gain (AUC \alg{0.62} vs. \alr{0.56} for cosine similarity)
\end{itemize}

\item \textbf{Empirical Validation}
\begin{itemize}
\item Grounds measurement in actual decisions
\item Reveals what skills \al{do} matter vs. what \al{should} matter
\end{itemize}
\end{itemize}
\end{block}
\end{column}

\begin{column}{0.48\textwidth}
\begin{block}{Potential Limitation}
\begin{itemize}
\item \textbf{Context Dependency}
\begin{itemize}
\item Reflects historical hiring patterns
\item Could perpetuate biases in past decisions
\item Mitigated by focus on predictive validity
\end{itemize}
\end{itemize}
\end{block}
\end{column}
\end{columns}
\end{frame}

\heading{Results}

\begin{frame}
\frametitle{Main Results: Predictive Power}
\begin{itemize}
\item \textbf{Our pipeline (with metric learning):} AUC = \alg{0.620}
\begin{itemize}
\item Significant predictive power from posting content
\end{itemize}

\item \textbf{SBERT embeddings with cosine similarity:} AUC = 0.562
\begin{itemize}
\item Raw embeddings capture some signal
\item Metric learning adds substantial value
\end{itemize}

\item \textbf{Random Forest on HR variables:} AUC = \alr{0.500}
\begin{itemize}
\item (tenure, experience, job level, proficiency ratings)
\item No predictive power beyond random guessing
\item Selection conditional on application explains limited power
\end{itemize}
\end{itemize}
\end{frame}

\begin{frame}
\frametitle{Selection Probability and Skill Distance}
\begin{center}
\fbox{\parbox{0.8\textwidth}{
\centering
\textbf{[INSERT FIGURE 4 FROM PAPER]} \\
Graph showing declining selection probability as skill distance increases
}}
\end{center}

\begin{block}{Key Finding}
\begin{itemize}
\item The probability of selection in the closest skill distance quintile is \alg{84\% higher} compared to the furthest quintile
\item Clear monotonic decline across quintiles
\item Robust to various controls and specifications
\end{itemize}
\end{block}
\end{frame}

\begin{frame}
\frametitle{Causal Evidence}
\begin{enumerate}
\item \textbf{Within-Vacancy Comparisons}
\begin{itemize}
\item Same vacancy, multiple applicants (261 cases)
\item \alg{70\%} of the time, candidate with shortest skill distance was selected
\item Conditional logit: \alr{35\% lower odds} per quintile increase in distance
\item Controls for vacancy-specific factors
\end{itemize}

\item \textbf{Within-Applicant Comparisons}
\begin{itemize}
\item Same applicant, multiple applications (33 cases)
\item \alr{50\% lower odds} for applications to more distant roles
\item Controls for applicant-specific traits
\end{itemize}
\end{enumerate}

\begin{block}{Implication}
Skill distance \al{causally} impacts selection, not just correlation
\end{block}
\end{frame}

\begin{frame}
\frametitle{Skill Distance Distribution}
\begin{center}
\fbox{\parbox{0.8\textwidth}{
\centering
\textbf{[INSERT FIGURE 5 FROM PAPER]} \\
Distribution of average skill distances across workers
}}
\end{center}

\begin{block}{Heterogeneity in Skill Alignment to Opportunity Landscape}
\begin{itemize}
\item Workers vary significantly in their average distance to available roles
\item This variation helps explain differences in:
\begin{itemize}
\item Application strategies
\item Success rates
\item Career mobility patterns
\end{itemize}
\end{itemize}
\end{block}
\end{frame}

\begin{frame}
\frametitle{Behavioral Insights}
\begin{columns}
\begin{column}{0.48\textwidth}
\begin{block}{High-distance applicants}
\begin{itemize}
\item Apply to \alg{1.5x more roles} (exploratory search)
\item One SD increase in average distance \so 0.61 SD increase in application distance
\item \alr{33.5\% lower} selection odds per application
\end{itemize}
\end{block}
\end{column}

\begin{column}{0.48\textwidth}
\begin{block}{Low-distance applicants}
\begin{itemize}
\item More targeted applications
\item Higher success rates
\item Often more passive in search behavior
\end{itemize}
\end{block}
\end{column}
\end{columns}

\vspace{1em}

\begin{block}{Actionable Insight}
Firms can reduce search frictions (targeted recommendations) or upskill (address gaps)
\end{block}
\end{frame}

\heading{Implications}

\begin{frame}
\frametitle{Benchmarking Against Traditional Metrics}
\begin{block}{Random Forest models showed:}
\begin{itemize}
\item Skill distance explained nearly all predictable variation in selection (AUC = \alg{0.60})
\item Traditional factors like tenure, experience, and performance ratings had \al{no predictive power} (AUC = \alr{0.50})
\end{itemize}
\end{block}

\begin{block}{Why Traditional Measures Underperform}
\begin{itemize}
\item Selection conditional on application
\item Skill alignment matters more than general qualifications
\item Text captures richer information than structured HR data
\end{itemize}
\end{block}
\end{frame}

\begin{frame}
\frametitle{Bridges Between Theory and Practice}
\begin{columns}
\begin{column}{0.48\textwidth}
\begin{block}{Theoretical Contributions}
\begin{itemize}
\item Validates job postings as containing meaningful skill signals
\item Shows skill alignment predicts outcomes better than traditional metrics
\item Addresses information asymmetry in internal labor markets
\end{itemize}
\end{block}
\end{column}

\begin{column}{0.48\textwidth}
\begin{block}{Practical Applications}
\begin{itemize}
\item Firms can quantify skill gaps to guide:
\begin{itemize}
\item Internal mobility recommendations
\item Targeted workforce development
\item Succession planning
\end{itemize}
\end{itemize}
\end{block}
\end{column}
\end{columns}

\vspace{1em}

\begin{center}
\al{Our method connects organizational theory to actionable insights}
\end{center}
\end{frame}

\begin{frame}
\frametitle{Case Study: Reducing Search Frictions}
\begin{center}
\fbox{\parbox{0.8\textwidth}{
\centering
\textbf{[INSERT VISUALIZATION]} \\
Simple diagram showing how skill distance measurement feeds into HR applications
}}
\end{center}

\begin{columns}
\begin{column}{0.3\textwidth}
\begin{block}{Challenge}
Employee struggles to identify relevant opportunities
\end{block}
\end{column}

\begin{column}{0.3\textwidth}
\begin{block}{Solution}
Skill distance analytics to:
\begin{itemize}
\item Recommend aligned roles
\item Highlight specific skills to develop
\item Design targeted learning paths
\end{itemize}
\end{block}
\end{column}

\begin{column}{0.3\textwidth}
\begin{block}{Outcome}
\begin{itemize}
\item Improved matching efficiency
\item Reduced search time
\item Increased mobility success rates
\end{itemize}
\end{block}
\end{column}
\end{columns}
\end{frame}

\begin{frame}
\frametitle{Methodological Innovation}
\begin{columns}
\begin{column}{0.48\textwidth}
\begin{block}{Prior Assumptions}
\begin{itemize}
\item Job postings are:
\begin{itemize}
\item Aspirational or generic
\item Best analyzed with one-size-fits-all models
\end{itemize}
\end{itemize}
\end{block}
\end{column}

\begin{column}{0.48\textwidth}
\begin{block}{Our Approach}
\begin{itemize}
\item Learns from real decisions what postings actually signal
\item Filters noise by focusing on dimensions that predicted outcomes
\item Provides framework for organization-specific adaptation
\end{itemize}
\end{block}
\end{column}
\end{columns}

\vspace{1em}

\begin{center}
\al{Framework generalizable to other domains requiring context-specific language interpretation}
\end{center}
\end{frame}

\begin{frame}
\frametitle{Conclusion}
\begin{itemize}
\item First validation against real decisions: Skill distance predicts selection
\item Outperforms traditional HR metrics (AUC \alg{0.62} vs. \alr{0.50}), resolves information asymmetry
\item \textbf{Nuance}: Informativeness is \al{maximized} by combining embeddings (ubiquitous) with firm-specific outcomes (novel)
\end{itemize}

\begin{block}{Opens New Lens on Labor Markets}
\begin{itemize}
\item Skill distance shapes search behavior and outcomes
\item Provides framework for reducing information asymmetries
\item Advances both theory and practice of skills-based approaches
\end{itemize}
\end{block}

\begin{block}{Impact}
Strengthens skills-first hiring, enables data-driven workforce strategies
\end{block}

\textbf{Future work}: Generalize framework to other domains
\end{frame}

\lastslide

\begin{frame}
\frametitle{Thank You!}
\begin{center}
\LARGE Questions?\\
\vspace{1cm}
\normalsize
Contact: \url{nikhilg@cmu.edu}
\end{center}
\end{frame}

\end{document}