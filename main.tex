\documentclass[aspectratio=169]{beamer}
\usepackage{simplified-presentation}
\usepackage{amsmath}

\title{How Informative are Job Posting Skill Measures?}
\subtitle{Evidence from Selection Decisions}
\author{Nikhil George \and Ramayya Krishnan \and Rahul Telang}
\institute{Carnegie Mellon University}
\date{}

\begin{document}

% Title slide
\begin{frame}
\titlepage
\end{frame}

% Motivation
\begin{frame}
\frametitle{Motivation}

\begin{itemize}
\item \textbf{Skills-first hiring revolution}
  \begin{itemize}
  \item Google, IBM, Walmart prioritize capabilities over credentials
  \item Reducing emphasis on traditional qualifications
  \end{itemize}
\vspace{0.5em}

\item \textbf{Job postings as unvalidated infrastructure}
  \begin{itemize}
  \item Powers algorithms, HR analytics, policy decisions
  \item Assumed to reflect skill demands, but \highlight{no validation against outcomes}
  \end{itemize}
\vspace{0.5em}

\item \textbf{Stakes are high}
  \begin{itemize}
  \item If uninformative: Undermines billion-dollar HR tech ecosystem
  \item If valid: Unlocks precision in labor markets
  \end{itemize}
\end{itemize}
\end{frame}

% The Core Gap
\begin{frame}
\frametitle{The Research Gap}

\begin{itemize}
\item \textbf{Central question:} Do skill measures derived from job postings predict actual hiring decisions?
\vspace{0.5em}

\item \textbf{Skepticism about job postings}
  \begin{itemize}
  \item May be aspirational or templated
  \item Could include inflated "nice-to-have" skills
  \end{itemize}
\vspace{0.5em}

\item \textbf{Critical validation needed}
  \begin{itemize}
  \item Research relies on posting data without outcome validation
  \item HR analytics industry depends on posting informativeness
  \end{itemize}
\end{itemize}
\end{frame}

% Research Questions
\begin{frame}
\frametitle{Research Questions}

\begin{enumerate}
\item Can skill distance measures built on job postings be validated against selection decisions?
\vspace{1em}

\item How informative are job postings relative to traditional HR metrics?
\vspace{1em}

\item Can analytics built on job postings reveal new insights about human capital management?
\end{enumerate}
\end{frame}

% Section divider
\sectionpage{Research Contribution}

% First Empirical Validation
\begin{frame}
\frametitle{First Empirical Validation of Job Posting Informativeness}

\begin{itemize}
\item \textbf{Novel approach}
  \begin{itemize}
  \item Skill distance metric derived solely from job text
  \item Validated against actual selection decisions
  \end{itemize}
\vspace{0.5em}

\item \textbf{Unique dataset}
  \begin{itemize}
  \item IT division of major financial firm
  \item 8,180 job postings, 1,370 internal applications
  \item Complete application histories
  \end{itemize}
\vspace{0.5em}

\item \textbf{Key findings}
  \begin{itemize}
  \item \positive{84\% higher} selection probability for closest skill matches
  \item \positive{70\%} of multi-applicant vacancies select closest candidate
  \item Job text outperforms traditional HR metrics (AUC \positive{0.62} vs. \negative{0.50})
  \end{itemize}
\end{itemize}
\end{frame}

% Literature Gap
\begin{frame}
\frametitle{A Gap in the Literature}

Prior work focuses on \highlight{extraction}, not \highlight{validation}:

\begin{itemize}
\item \textbf{Taxonomy-based approaches} (O*NET, ESCO)
  \begin{itemize}
  \item Static categories, misses contextual nuance
  \end{itemize}
\vspace{0.5em}

\item \textbf{Collaborative filtering} (LinkedIn)
  \begin{itemize}
  \item Co-occurrence ≠ predictive validity
  \end{itemize}
\vspace{0.5em}

\item \textbf{Credential proxies}
  \begin{itemize}
  \item Losing relevance in skills-first hiring
  \end{itemize}
\end{itemize}

\begin{block}{Our Breakthrough}
Outcome-grounded validation: 
Not just what skills are listed, but which ones actually matter
\end{block}
\end{frame}

% Research Setting
\begin{frame}
\frametitle{Research Setting \& Data}

\begin{itemize}
\item \textbf{Internal labor market as ideal laboratory}
  \begin{itemize}
  \item Workers move across roles within firm
  \item Can link current job to positions sought
  \item Observe both successful and unsuccessful applications
  \end{itemize}
\vspace{0.8em}

\item \textbf{Dataset overview}
  \begin{itemize}
  \item IT division, technology roles with precise requirements
  \item 8,180 job postings (2018–2021)
  \item 1,370 internal applications (650 selected, 720 rejected)
  \end{itemize}
\end{itemize}

\begin{block}{Key Advantage}
Ability to measure skill distance between current and target roles
\end{block}
\end{frame}

% Sample Job Comparison
\begin{frame}
\frametitle{Sample Job Postings Comparison}

\begin{center}
\textbf{Target Position: Senior Data Architect}
\end{center}

\begin{columns}
\begin{column}{0.48\textwidth}
\begin{block}{Close Match (Selected)}
\begin{itemize}
\item \textbf{Position:} Data Engineer
\item \textbf{Shared skills:} Data modeling, ETL, SQL
\item \textbf{Distance drivers:} Limited architecture experience
\end{itemize}
\end{block}
\end{column}

\begin{column}{0.48\textwidth}
\begin{block}{Distant Match (Not Selected)}
\begin{itemize}
\item \textbf{Position:} Full-Stack Developer
\item \textbf{Missing skills:} Data warehousing, dimensional modeling
\item \textbf{Different focus:} UI/UX vs. backend data
\end{itemize}
\end{block}
\end{column}
\end{columns}
\end{frame}

% Conceptual Framework
\begin{frame}
\frametitle{Conceptual Framework}

\begin{itemize}
\item Job descriptions encode skill requirements that matter for selection
\item Selection decisions reveal which aspects of postings are most important
\item Skill alignment between current and target roles predicts outcomes
\end{itemize}

\begin{block}{Key Insight}
When an applicant seeks a position that is "distant" in required skills from their current job, they have lower probability of being selected
\end{block}

\begin{center}
\begin{tabular}{|l|c|}
\hline
\textbf{Applicant-Vacancy Skill Distance} & \textbf{Selection Probability} \\
\hline
Low (close match) & High \\
Medium & Medium \\
High (distant match) & Low \\
\hline
\end{tabular}
\end{center}
\end{frame}

% Section divider
\sectionpage{Methodology}

% Methodological Pipeline
\begin{frame}
\frametitle{Methodological Approach}

\begin{enumerate}
\item \textbf{Job Posting Embeddings}
   \begin{itemize}
   \item SBERT contextual language model captures semantic richness
   \item 384-dimensional vectors represent each job posting's content
   \end{itemize}
\vspace{0.5em}

\item \textbf{Dimension Reduction}
   \begin{itemize}
   \item PCA + Fuzzy C-Means clustering reduces to manageable dimensions
   \item Preserves key semantic information while removing noise
   \end{itemize}
\vspace{0.5em}

\item \textbf{Distance Metric Learning}
   \begin{itemize}
   \item Reweights dimensions to reflect firm's historical hiring priorities
   \item Stretches dimensions that differentiate selected/rejected candidates
   \end{itemize}
\vspace{0.5em}

\item \textbf{Validation}
   \begin{itemize}
   \item AUC 0.62 on held-out data using learned metric
   \end{itemize}
\end{enumerate}
\end{frame}

% Metric Learning
\begin{frame}
\frametitle{Metric Learning: Adapting to Context}

\begin{columns}
\begin{column}{0.5\textwidth}
\begin{itemize}
\item Distance between jobs $j_c$ and $j_v$:
\begin{align*}
d(j_c, j_v; M) = \sqrt{(j_c - j_v)^T M (j_c - j_v)}
\end{align*}

\item Learn matrix $M$ that:
\begin{itemize}
\item Maximizes distance between non-selected pairs
\item Minimizes distance between selected pairs
\end{itemize}

\item Reweights skill dimensions based on what the firm historically valued
\end{itemize}
\end{column}

\begin{column}{0.5\textwidth}
\begin{block}{Strength: Domain Adaptation}
Generic embeddings (cosine similarity) treat all skills equally,\\
metric learning \highlight{tailors to firm context}
\end{block}

\begin{block}{Limitation: Context Dependency}
Reflects historical patterns, could perpetuate biases
\end{block}
\end{column}
\end{columns}
\end{frame}

% Section divider
\sectionpage{Results}

% Main Results
\begin{frame}
\frametitle{Predictive Power of Job Posting Content}

\begin{center}
\begin{tabular}{|l|c|l|}
\hline
\textbf{Model} & \textbf{AUC} & \textbf{Interpretation} \\
\hline
Our metric learning approach & \positive{0.620} & Significant predictive power \\
SBERT with cosine similarity & 0.562 & Some signal in raw embeddings \\
HR variables (tenure, experience) & \negative{0.500} & No predictive power \\
\hline
\end{tabular}
\end{center}
\vspace{1em}

\begin{block}{Key Finding}
Job posting content has significant predictive power, outperforming traditional HR variables
\end{block}

\begin{itemize}
\item Metric learning adds substantial value over generic embeddings
\item Selection conditional on application explains limited power of HR variables
\end{itemize}
\end{frame}

% Selection Probability
\begin{frame}
\frametitle{Selection Probability and Skill Distance}

\begin{center}
\begin{tabular}{|l|c|}
\hline
\textbf{Skill Distance Quintile} & \textbf{Relative Selection Probability} \\
\hline
Q1 (Closest) & \positive{+84\%} \\
Q2 & \positive{+52\%} \\
Q3 & Reference \\
Q4 & \negative{-25\%} \\
Q5 (Furthest) & \negative{-46\%} \\
\hline
\end{tabular}
\end{center}
\vspace{1em}

\begin{block}{Key Result}
Selection probability in closest skill distance quintile is \positive{84\% higher} compared to furthest quintile
\end{block}
\vspace{0.5em}

\begin{itemize}
\item Clear monotonic decline across quintiles
\item Robust to various controls and specifications
\end{itemize}
\end{frame}

% Causal Evidence
\begin{frame}
\frametitle{Causal Evidence from Quasi-Experiments}

\begin{columns}
\begin{column}{0.48\textwidth}
\begin{block}{Within-Vacancy Comparisons}
\begin{itemize}
\item 261 vacancies with multiple applicants
\item \positive{70\%} select candidate with shortest distance
\item \negative{35\% lower odds} per quintile increase in distance
\end{itemize}
\end{block}
\end{column}

\begin{column}{0.48\textwidth}
\begin{block}{Within-Applicant Comparisons}
\begin{itemize}
\item 33 applicants with multiple applications
\item \negative{50\% lower odds} for more distant roles
\item Controls for individual traits
\end{itemize}
\end{block}
\end{column}
\end{columns}
\vspace{1em}

\begin{center}
\begin{tabular}{|l|c|}
\hline
\textbf{Design} & \textbf{Controls For} \\
\hline
Within-Vacancy & Vacancy-specific factors (role, team, timing) \\
Within-Applicant & Applicant-specific factors (quality, experience) \\
\hline
\end{tabular}
\end{center}

\begin{center}
Skill distance \highlight{causally} impacts selection
\end{center}
\end{frame}

% Behavioral Insights
\begin{frame}
\frametitle{Skill Distance Shapes Job Search Behavior}

\begin{columns}
\begin{column}{0.48\textwidth}
\begin{block}{High-distance Applicants}
\begin{itemize}
\item Apply to \positive{1.5x more roles}
\item More exploratory search
\item \negative{33.5\% lower} selection odds
\end{itemize}
\end{block}
\end{column}

\begin{column}{0.48\textwidth}
\begin{block}{Low-distance Applicants}
\begin{itemize}
\item More targeted applications
\item Higher success rates
\item Often more passive in search
\end{itemize}
\end{block}
\end{column}
\end{columns}
\vspace{1em}

\begin{center}
\begin{tabular}{|l|c|c|}
\hline
\textbf{Measure} & \textbf{High-Distance Workers} & \textbf{Low-Distance Workers} \\
\hline
Application Volume & \positive{Higher (1.5x)} & Lower \\
Application Distance & More variable & Less variable \\
Selection Rate & \negative{Lower} & Higher \\
\hline
\end{tabular}
\end{center}
\end{frame}

% Section divider
\sectionpage{Implications}

% Theory & Practice
\begin{frame}
\frametitle{Bridging Theory and Practice}

\begin{columns}
\begin{column}{0.48\textwidth}
\begin{block}{Theoretical Contributions}
\begin{itemize}
\item Validates job postings as informative signals
\item Advances skills-based labor market theories
\item Addresses information asymmetry
\end{itemize}
\end{block}
\end{column}

\begin{column}{0.48\textwidth}
\begin{block}{Practical Applications}
\begin{itemize}
\item Internal mobility recommendations
\item Targeted skill development
\item Evidence-based succession planning
\end{itemize}
\end{block}
\end{column}
\end{columns}
\vspace{1em}

\begin{center}
\begin{tabular}{|l|l|}
\hline
\textbf{Research Domain} & \textbf{Contribution} \\
\hline
Labor Economics & Validates key data source, quantifies skill distance \\
HR Analytics & Provides framework for mobility analysis \\
NLP/ML & Demonstrates domain adaptation of language models \\
\hline
\end{tabular}
\end{center}
\end{frame}

% Methodological Innovation
\begin{frame}
\frametitle{Methodological Innovation}

\begin{block}{Prior Assumptions}
Job postings are aspirational or generic, best analyzed with one-size-fits-all models
\end{block}

\begin{block}{Our Approach}
Learn from real selection decisions what aspects of postings actually matter
\end{block}
\vspace{1em}

\begin{center}
\begin{tabular}{|l|l|}
\hline
\textbf{Challenge} & \textbf{Our Solution} \\
\hline
Generic embeddings miss context & Metric learning adapts to organization \\
"Noise" in job descriptions & Selection decisions reveal what matters \\
Historical biases in hiring & Focus on predictive validity, not prescription \\
\hline
\end{tabular}
\end{center}
\vspace{1em}

Framework is generalizable to other domains requiring
context-specific language interpretation
\end{frame}

% Conclusion
\begin{frame}
\frametitle{Conclusion}

\begin{itemize}
\item First validation against real decisions: Job postings are highly informative
\item Outperforms traditional HR metrics (AUC \positive{0.62} vs. \negative{0.50})
\item Informativeness maximized by combining generic embeddings with firm-specific outcomes
\end{itemize}

\begin{block}{Opens New Lens on Labor Markets}
\begin{itemize}
\item Skill distance shapes search behavior and outcomes
\item Enables data-driven workforce strategies
\item Strengthens case for skills-first hiring approaches
\end{itemize}
\end{block}

\begin{center}
\highlight{Our validation transforms job postings from static text into a dynamic, validated signal for labor market decisions}
\end{center}
\end{frame}

% Thank You
\begin{frame}
\frametitle{Thank You!}
\begin{center}
\LARGE Questions?\\
\vspace{1cm}
\normalsize
Contact: nikhilg@cmu.edu
\end{center}
\end{frame}

\end{document}